\documentclass[a4paper,oneside,11pt]{scrartcl}

\usepackage[ngerman]{babel}
\usepackage[T1]{fontenc}
\usepackage[utf8]{inputenc}
\usepackage[bookmarksopen]{hyperref}
\usepackage[dvips]{color}
\usepackage{lmodern}
\usepackage{scrpage2}

\newcommand{\titel}{Rubik's Cube-Workshop Cheat Sheet}
\newcommand{\autor}{Jan Dillmann (\href{http://www.jandillmann.de}{www.jandillmann.de})}
\newcommand{\datum}{\today}

\newcommand{\grad}{$^\circ$}

% Serifenlose Schrift
\renewcommand{\familydefault}{\sfdefault}
% kein Absatzeinzug
\setlength{\parindent}{0mm}
% keine Seitenzahlen
\clearscrplain
\pagestyle{scrplain}

\hypersetup{%
	pdftitle={\titel},%
	pdfauthor={\autor},%
	colorlinks=true,%
	linkcolor=black,%
	urlcolor=black
}

\title{\titel}
\author{\autor}
\date{\datum}


\begin{document}
\maketitle

\section*{Seiten} % (fold)
\label{sec:seiten}

\textbf{\underline{U}}p,
\textbf{\underline{D}}own,
\textbf{\underline{L}}eft,
\textbf{\underline{R}}ight,
\textbf{\underline{F}}ront,
\textbf{\underline{B}}ack

% section seiten (end)

\section*{Notation} % (fold)
\label{sec:notation}

$R^{-1}D^{-1}RD \,\Rightarrow$ \glqq Right inverted, down inverted, right, down\grqq \\\glqq Rechte Seite um 90\grad{} gegen den Uhrzeigersinn drehen, untere Seite um 90\grad{} gegen den Uhrzeigersinn drehen, rechte Seite um 90\grad{} im Uhrzeigersinn drehen, untere Seite um 90\grad{} im Uhrzeigersinn drehen.\grqq

% section notation (end)

\section*{Algorithmen} % (fold)
\label{sec:algorithmen}

\subsubsection*{Ebene 1} % (fold)
\label{ssub:ebene_1}

\begin{description}
	\item[Kanten drehen] $F^{-1}UL^{-1}U^{-1}$
	\item[Ecken positionieren] $R^{-1}D^{-1}RD$
\end{description}

% subsubsection ebene_1 (end)

\subsubsection*{Ebene 2} % (fold)
\label{ssub:ebene_2}

\begin{description}
	\item[Kante nach rechts kippen] $URU^{-1}R^{-1}U^{-1}F^{-1}UF$
	\item[Kante nach links kippen] $U^{-1}L^{-1}ULUFU^{-1}F^{-1}$
\end{description}

% subsubsection ebene_2 (end)

\subsubsection*{Ebene 3} % (fold)
\label{ssub:ebene_3}

\begin{description}
	\item[Kreuz auf der Oberseite] $FRUR^{-1}U^{-1}F^{-1}$
	\item[Mittelstücke der Kanten positionieren] $RUR^{-1}URU^2R^{-1}U$
	\item[Ecken an die richtige Stelle bringen] $URU^{-1}L^{-1}UR^{-1}U^{-1}L$
	\item[Ecken positionieren] $R^{-1}D^{-1}RD$
\end{description}

% subsubsection ebene_3 (end)

% section algorithmen (end)

\end{document}